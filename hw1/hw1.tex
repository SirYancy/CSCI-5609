Q1: What data are you most surprised to find are available via the portal or what data are you most interested to explore?

The one that raised my eyebrow the most looking through it is the Vacant Buildings dataset. I think just knowing how many buildings sit empty can give us a better understanding of how we can more efficiently use land. I actually want to see if there's any correlation between vacant buildings and other indicators like income level in the area. It looks like the number of vacant houses skyrocketed (surprise!) around 2008.

Q2: For the budget, identify three interesting questions you think it might be possible to answer by looking at the data.  Try to make these questions non-obvious.  For example, it's not all that interesting to just ask for a total, like "what is the total amount the city will spend on employee salaries for the impound lot in 2019".  We can look that answer up pretty easily in a table.  It would be much more interesting to ask a question that involves data analysis like comparison, tracking over time, aggregation by category, etc.  For each question, comment on the extent to which the visualizations currently available on the website could help you answer the question.

1. First of all, at the risk of outing my politics, I'd like to point out just how little money the city spends on WIC each year. And people complain about welfare being so expensive. Compared to police, it's literally nothing.

2. What city employee positions have the best salaries? The data visualizations as set up do not tell us much about this. However, the data is there. We know how much they spend on salary and fringe in each department. We do not know, necessarily, how many employees that includes, but it might be buried somewhere. But it would be interesting to know where the good government jobs are.

3. It's interesting to see all of the different economic development programs that are mentioned in the budget. I was wondering how the costs of those programs have evolved over time, and was easily able to bring up a line chart that shows just that.

Q3: For the budget, identify three questions you think are not possible to answer by looking at the data alone, as in you probably would need some more context from another dataset or from actual people in order to answer these type of questions.  For each of these questions, identify what other information would allow you to answer the question and where that might come from (e.g., the mayor's office, St. Paul residents, state government, students, other cities).

1. In the expense budget, the largest single line item is salaries and wages. The largest piece of that, is public safety, and the largest piece of that is the police, specifically an item labeled Patrol Operations, which I assume means beat cops. I'd like to see a heat map of where those patrols are prioritized. And where they're not. I suppose the St Paul PD would have some data about this.

2. I get that Sewer Utility is a big deal, but I never realized just to what extent it was a major source of spending and funding. It would be interesting to talk to someone about just to what extent the city's economy is tied up in dealing with...this issue.

3. One of the issues that really fascinates me about city administration is gentrification. It looks like $118 million were collected in property taxes. I want to see a map of what property that tax was collected from. Sort of a look at the property value distribution of all of the city's real estate. Where are taxes high, where are they low? That sort of thing.

Q4: We will get to meet the open data technical team that maintains the portal.  What technical questions do you have for them?  These might revolve around things like how they have setup the website, their mission, what they hope to do in the future, and/or what challenges they have faced?

A friend of mine works in record-keeping for a school district. And one of the most difficult tasks he has is getting clean, usable data from hundreds of teachers, counselors, principals, and paraprofessionals. What incentives (carrot or stick) are in place to ensure that various goups and organizations from around the city are properly motivated to submit proper data, be thorough and accurate in their record-keeping, and keep doing it year after year.

Also, how does the record keeping change if a field is added to a particular dataset one year? Or removed (I would imagine the latter is rare)?

Q5: We will also get to meet the mayor and his budget team.  We know they are especially interested in viewing the budget as a reflection of the values of the city.  If we are going to create some data visualizations that relate to this concept of "budget reflecting values", then we need to know more about what this means to the mayor and how this high-level idea connects to actual numbers.  What questions should we ask him and his budget team to try to clarify this and begin to determine the types of visualizations that could actually help with policy making?

Honestly, I wouldn't mind going through it line by line to determine what some items are. For example, if you take the entire Expense Budget broken down by Item Detail, the second item is "Transfers Out Operations". I have no idea what that means, or why it costs $60 million. It looks like it's just moving money around, but does it actually cost money to do that? To what extent does this need to be such a massive piece of the budget? So, putting together a list of all of the items that are not clear, jargony, or esoteric, and asking for them to be clarified would be very valuable. 
